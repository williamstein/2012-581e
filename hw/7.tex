\documentclass{article}
%\voffset=-.12\textheight
%\textheight=1.24\textheight
\include{macros}
\usepackage{url}
\usepackage{color}
\definecolor{dblackcolor}{rgb}{0.0,0.0,0.0}
\definecolor{dbluecolor}{rgb}{.01,.02,0.7}
\definecolor{dredcolor}{rgb}{0.6,0,0}
\definecolor{dgraycolor}{rgb}{0.30,0.3,0.30}
\definecolor{dgreen}{rgb}{0,0.3,0}
\usepackage{listings} 
\usepackage{hyperref} 
\lstdefinelanguage{Sage}[]{Python}
{morekeywords={True,False,sage,singular},
sensitive=true}
\lstset{frame=none,
          showtabs=False,
          showspaces=False,
          showstringspaces=False,
          commentstyle={\ttfamily\color{dredcolor}},
          keywordstyle={\ttfamily\color{dbluecolor}\bfseries},
          stringstyle ={\ttfamily\color{dgraycolor}\bfseries},
          language = Sage,
	  basicstyle={\scriptsize \ttfamily},
	  aboveskip=.3em,
	  belowskip=.1em
          }

\title{\dred{Math 581e, Fall 2012, Homework 7}}
\author{William Stein ({\tt wstein@uw.edu})}
\date{Due: Friday, November 16, 2012}

\begin{document}

\maketitle

{\color{dbluecolor}There are \ref{last} problems. Turn your solutions
  in Friday, November 16, 2012 in class.  You may work with other
  people and can find the \LaTeX{} of this file at
  \url{http://wstein.org/edu/2012/ant/hw/}.  If you use Sage to solve
  a problem, include your code in your solution. I have office hours
  12:30--2:00 on Wednesdays in Padelford C423.  }

For any Dedekind domain $R$ at all, define the {\em class group} $\Cl(R)$ to be
the group of fractional ideals modulo the subgroup of principal
fractional ideals.  This definition makes sense for an arbitrary
Dedekind domain.

{\bf\color{dredcolor} Warning:} I just made up all of these problems
from scratch, so if something seems wrong or impossible, ask me!

\begin{enumerate}

\item Let $\cO_K$ be the ring of integers of a number field and let
  $n$ be a positive integer.  
  \begin{enumerate}
    \item Prove that $R=\cO_K[\frac{1}{n}]$ is a Dedekind domain.  
    \item Prove that $\Cl(R)$ is finite.
    \item Describe (with proof) a useful relationship between $\Cl(R)$ and $\Cl(\cO_K)$.
  \end{enumerate}

\item Let $R=k[t]$, where $k$ is an algebraically closed field.  Of course,
$R$ is a Dedekind domain.
  \begin{enumerate}
  \item Prove that $\Cl(R)$ is trivial (of order $1$).
  \item Prove that the group $U_R$ of units in $R$ is not finitely generated.
  \end{enumerate}

\item  Let $k$ be a {\em finite field} of characteristic $\neq 2$, and consider
the Dedekind domain $R=k[x,y]/(y^2-x^3-x)$.  [You can use anything you know
from outside of class from algebra or algebraic geometry on this problem.]
\begin{enumerate}
\item Let $I\subset R$ be a nonzero ideal.  Prove that the norm $N(I) = \#(R/I)$
is finite.
\item Let $B$ be a positive integer.  Prove that there are finitely
many nonzero ideals $I$ of $R$ such that $N(I)\leq B$.
\item Prove that the unit group $U_R = R^*$ is a finite cyclic group.
\item Nonetheless, prove that $\Cl(R)$ is {\em infinite}. [[In fact, this part of the problem is completely {\em wrong} -- the class group is in bijection with the group of rational points on the elliptic curve over the finite field, which is finite.]]
\end{enumerate}

\item \label{last}\begin{enumerate}
\item Prove that if $K$ is any number field, then the torsion subgroup of the group
$U_K=\cO_K^*$ has even order.
\item Prove that if $K=\Q(\sqrt{D})$, with $D\leq -1$ square free, is
  a quadratic imaginary field, then the unit group $U_K$ of $\cO_K$ has
  order $2$, $4$, or $6$.
\item Prove that if $K$ is a number field of odd degree, then the
  torsion subgroup of $U_K$ has order $2$
\item What is the torsion subgroup of the unit group of the $389$th
  cyclotomic field $\Q(\zeta_{389})$.
\end{enumerate}

\end{enumerate}

\end{document}

\documentclass{article}
%\voffset=-.12\textheight
%\textheight=1.24\textheight
\include{macros}
\usepackage{url}
\usepackage{color}
\definecolor{dblackcolor}{rgb}{0.0,0.0,0.0}
\definecolor{dbluecolor}{rgb}{.01,.02,0.7}
\definecolor{dredcolor}{rgb}{0.6,0,0}
\definecolor{dgraycolor}{rgb}{0.30,0.3,0.30}
\definecolor{dgreen}{rgb}{0,0.3,0}
\usepackage{listings} 
\usepackage{hyperref} 
\lstdefinelanguage{Sage}[]{Python}
{morekeywords={True,False,sage,singular},
sensitive=true}
\lstset{frame=none,
          showtabs=False,
          showspaces=False,
          showstringspaces=False,
          commentstyle={\ttfamily\color{dredcolor}},
          keywordstyle={\ttfamily\color{dbluecolor}\bfseries},
          stringstyle ={\ttfamily\color{dgraycolor}\bfseries},
          language = Sage,
	  basicstyle={\scriptsize \ttfamily},
	  aboveskip=.3em,
	  belowskip=.1em
          }

\title{\dred{Math 581e, Fall 2012, Homework 8}}
\author{William Stein ({\tt wstein@uw.edu})}
\date{Due: Friday, November 30, 2012}

\begin{document}

\maketitle

{\color{dbluecolor}  This is the last homework assignment.  There are \ref{last} problems. Turn your solutions
  in Friday, November 30, 2012 in class.  You may work with other
  people and can find the \LaTeX{} of this file at
  \url{http://wstein.org/edu/2012/ant/hw/}.  If you use Sage to solve
  a problem, include your code in your solution. I have office hours
  12:30--2:00 on Wednesdays in Padelford C423.  }


\begin{enumerate}

\item Turn in rough draft of your project.

\item Let $K$ be the cubic field obtained by adjoining a root of $f=$ to $\Q$.
\begin{enumerate}
\item Show that $r=3, s=0$, i.e., there are 3 real embeddings.
\item Compute (e.g., using Sage) explicit generators for the unit group $U_K$.
\item Draw a picture that illustrates how $U_K$ maps to a lattice in a codimension one subspace of $\R^3$.
\item Choose a basis for the image of $U_K$, and compute the $2\times 2$ matrix $A$ corresponding to the
dot product pairing on that basis.
\item Compute the absolute value of the determinant of $A$, which is
  (basically) a quantity called the {\em regulator} of the number
  field $K$.
\end{enumerate}

\item \label{last} (This problem is inspired by Aly's talk about pseudo-basis, and is Lemma~1.2.20 of \url{http://wstein.org/5canz/craig/math/Cohen\%20--\%20Advanced\%20topics\%20in\%20computational\%20number\%20theory.pdf}) Let
  $\mathfrak{a}$ and $\mathfrak{b}$ be nonzero ideals of a Dedekind domain $R$.  Prove that the
  $R$-modules $\mathfrak{a}\oplus \mathfrak{b}$ and $R\oplus
  \mathfrak{a}\mathfrak{b}$ are isomorphic, as follows:
\begin{enumerate}
\item Reduce to the case that $\mathfrak{a}$ and $\mathfrak{b}$ are
  integral ideals, by using that $\mathfrak{a}\isom
  \mathfrak{a}(\alpha)$ for nonzero $\alpha$.
\item Use the lemma we proved ``there is $\alpha$ such that $\alpha
  I^{-1}\subset R$ is coprime to $J$'' to reduce to the case when
  $\mathfrak{a}$ and $\mathfrak{b}$ are coprime.
\item Define a map $f:\mathfrak{a}\oplus \mathfrak{b} \to R$ by
  $f(a,b) = a-b$. Show that $f$ is an $R$-module homomorphism, then use
  that $\mathfrak{a}$ and $\mathfrak{b}$ are coprime to deduce that we
  have an exact sequence
$$
  0 \to \mathfrak{a}\cap \mathfrak{b} \to \mathfrak{a}\oplus \mathfrak{b} \to R \to 0.
$$
\item Understand this: Since $R$ is a free module, it is projective,
  which implies that the above exact sequence splits, which proves the
  statement.
\end{enumerate}

\end{enumerate}

\end{document}

\documentclass{article}
%\voffset=-.12\textheight
%\textheight=1.24\textheight
\include{macros}
\usepackage{url}
\usepackage{color}
\definecolor{dblackcolor}{rgb}{0.0,0.0,0.0}
\definecolor{dbluecolor}{rgb}{.01,.02,0.7}
\definecolor{dredcolor}{rgb}{0.6,0,0}
\definecolor{dgraycolor}{rgb}{0.30,0.3,0.30}
\definecolor{dgreen}{rgb}{0,0.3,0}
\usepackage{listings} 
\usepackage{hyperref} 
\lstdefinelanguage{Sage}[]{Python}
{morekeywords={True,False,sage,singular},
sensitive=true}
\lstset{frame=none,
          showtabs=False,
          showspaces=False,
          showstringspaces=False,
          commentstyle={\ttfamily\color{dredcolor}},
          keywordstyle={\ttfamily\color{dbluecolor}\bfseries},
          stringstyle ={\ttfamily\color{dgraycolor}\bfseries},
          language = Sage,
	  basicstyle={\scriptsize \ttfamily},
	  aboveskip=.3em,
	  belowskip=.1em
          }

\title{\dred{Math 581e, Fall 2012, Homework 4}}
\author{William Stein ({\tt wstein@uw.edu})}
\date{Due: Friday, October 26, 2012}

\begin{document}

\maketitle

{\color{dbluecolor}There are \ref{last} problems. Turn your solutions
  in Friday, October 26, 2012 in class.  You may work with other
  people and can find the \LaTeX{} of this file at
  \url{http://wstein.org/edu/2012/ant/hw/}.  If you use Sage to solve
  a problem, include your code in your solution. I have office hours
  12:30--2:00 on Wednesdays in Padelford C423.  }

\begin{enumerate}

\item Let $I$ be a nonzero integral ideal in the ring of integers
  $\cO_K$ of a number field $K$.  {\color{dgreen} Prove that $I^{-1}=\{\alpha\in K:
  \alpha I \subset I\}$. } \hint{In class, we proved exactly this
    statement in the case when $I$ is a prime ideal.}

\item {\color{dgreen} Give the most elementary proof you can that the abelian group of
  fractional ideals in the ring of integers of a number field is
  torsion free.} \hint{Obviously, this follows from the theorem that
    every fractional ideal can be written uniquely as a product of
    prime ideals.  However, that theorem took a lot of work to prove.
    Find an argument that uses less total ``proof energy''.}

\item Explicitly factor the fractional ideal $(2/3)\cO_K$ for $K$
each of the following fields: $K=\Q, \Q(\sqrt{5}), \Q(\sqrt{2}), \Q(\sqrt[3]{2}), \Q(\sqrt{2},\sqrt{3})$.
\hint{You can use Sage if you want.}

\item\label{last}
{\color{dgreen} Come up with an idea for what you might do your final project on.}
(You can change based on my feedback, etc.)
\begin{enumerate}
\item Title of your final project:
\item Abstract of your final project (one paragraph):
\item Other people you might collaborate with on your final project (not required).
\item Estimated amount of time you intend to spend on your final project.
\end{enumerate}

\end{enumerate}

\end{document}
